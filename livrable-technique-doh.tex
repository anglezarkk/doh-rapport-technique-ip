\documentclass[a4paper,12pt]{article}
\usepackage[utf8]{inputenc}
\usepackage{fancyhdr}
\usepackage{enumitem}
\usepackage{hyperref}
\usepackage{natbib}
\usepackage[french]{babel}

\setlength{\parindent}{1cm}
\usepackage{mwe,lipsum}
\usepackage[section]{placeins}
\usepackage{dblfloatfix}
\usepackage{float}
\usepackage[utf8]{inputenc}
\usepackage[T1]{fontenc}
\usepackage{graphicx}
\usepackage{fullpage}
\usepackage{eso-pic}
\usepackage{layout}
\usepackage{indentfirst}
\usepackage{siunitx}
\usepackage{amsmath}
\usepackage{amssymb}
\usepackage{tabularx}
\usepackage{multicol}
\usepackage{multirow}

\usepackage{caption}
\usepackage{subcaption}

\setlength{\headsep}{0.3cm}
\renewcommand*\contentsname{Table des matières}

\begin{document}
	\pagestyle{fancy}
	\headheight=15pt
	\fancyhf{}
	\renewcommand{\footrulewidth}{0.4pt}
	
	\lhead{DNS over HTTPS}
	\rhead{IP}
	% \lfoot{Equipe 9 - 2020}
	\rfoot{\thepage}
	
	%   TITLE PAGE
	\begin{titlepage}
		
		\begin{center}
			\rule{\textwidth}{1pt} % Thick horizontal rule
			
			\vspace{2pt}\vspace{-\baselineskip} % White space between rules
			
			\rule{\textwidth}{0.4pt} % Thin horizontal rule
			
			\vspace{0.1\textheight} % White space between the top rules and title
			
			%------------------------------------------------
			%	Title
			%------------------------------------------------
			
			\textcolor{black} {
				{\Huge DNS over HTTPS}\\[0.5\baselineskip]
			}
			
			\vspace{0.01\textheight} % White space between the title and short horizontal rule
			
			\rule{0.3\textwidth}{0.4pt} % Short horizontal rule under the title
			
			\vspace{0.1\textheight}
			
			%------------------------------------------------
			%	Author
			%------------------------------------------------
			
			{\Large \textsc{Arnaud Lombardi - Théo Grosperrin}} % Author name
			
			\vspace{0.025\textheight}
			
			{\large \textsc{\today}}
			
			\vfill % White space between the author name and publisher
			
		\end{center}
		
		%------------------------------------------------
		%	Publisher
		%------------------------------------------------
		
		\centering
		{\includegraphics[scale=0.5]{Images/INSA_LOGO.png}}\\[1.7\baselineskip]
		{\includegraphics[scale=0.18]{Images/TC.jpg}}
		
		\vspace{0.1\textheight} % White space under the publisher text
		
		%------------------------------------------------
		%	Bottom rules
		%------------------------------------------------
		
		\centering
		
		\rule{\textwidth}{0.4pt} % Thin horizontal rule
		
		\vspace{2pt}\vspace{-\baselineskip} % White space between rules
		
		\rule{\textwidth}{1pt} % Thick horizontal rule
		
	\end{titlepage}
	
	% Front matter %
	\setcounter{page}{1}
	\pagenumbering{arabic}
	
	%------------------------------------------------
	%	Table of contents
	%------------------------------------------------
	\hspace{2cm}
	\tableofcontents
	
	\newpage
	
	\section{En quelques mots, qu'est que le DoH ?}
	Il s'agit de la résolution des requêtes DNS en utilisant le protocole HTTPS, au lieu de simples requêtes en clair sur le port 53 (DNS classique). L'objectif de cette méthode est d'augmenter la sécurité du cote utilisateur : cela permet d'améliorer la vie privée de l'utilisateur (le trafic entre le client et le serveur DNS est chiffré, illisible par les machines se trouvant sur le chemin), et d'éviter les attaques de type "man-in-the-middle".
	
	\section{Détails techniques}
	Le DoH est un standard publié par l'IETF : \href{https://tools.ietf.org/html/rfc8484}{RFC 8484}.
	
	\subsection{Le fonctionnement du DNS et ses limites}
	
	\subsection{Le fonctionnement du DoH}
	
	\section{Les différentes implémentations possibles}
	
	\subsubsection{DoH natif au sein du navigateur}
	
	
	\subsubsection{DoH proxy sur le réseau local}
	
	Dans ce scénario, les machines du réseau local effectuent des requêtes DNS traditionnelles sur le port 53, vers un serveur DNS installé sur le réseau local. Ce serveur DNS effectue ensuite une requête DNS récursive vers un serveur DNS compatible DoH. Ainsi, la requête est chiffrée lorsqu'elle sort du LAN.
	
	\begin{figure}[H]
		\begin{center}
			{\includegraphics[scale=0.6]{Images/schema_doh_proxy_lan.png}}
		\end{center}
		\caption{Schéma de l'implémentation d'un serveur DoH proxy sur un réseau local.}
	\end{figure}
	
	On constate bien l'inconvénient de ce système : les requêtes entre les clients et le serveur local ne sont pas sécurisées, et on se base sur le fait que l'on fait confiance au serveur proxy.
	
	\subsubsection{DoH proxy sur système local}
	
	\section{Mise en contexte}
	
	\subsection{Support logiciel}
	
	\subsubsection{Système d'exploitation}
	
	\subsubsection{Navigateurs compatibles}
	
	\subsubsection{Serveurs compatibles}
	
	\subsection{Critiques}
	
	\section{Les problèmes que peuvent poser le DoH}
	
	
	
\end{document}